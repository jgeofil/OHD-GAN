\footnotesize
\tcbset{enhanced, before skip=1cm, nobeforeafter, width=0.5\linewidth}
\begin{tcolorbox}[
    arc=0mm, 
    colback=cadmiumgreen!10!white, 
    coltext=cadmiumgreen!90!black,  
    colframe=cadmiumgreen!90!black,
    colbacktitle=cadmiumgreen!80,
    leftrule=3mm,
    rightrule=0mm, 
    toprule=0mm, 
    bottomrule=0mm, 
    box align=top]
    
The architecture of \gls{ohd}-GAN should be engineered to match the data, not the other way around. Data with minimal transformations, to the extent possible. In addition to preventing information loss, this ensures models will reflect the real generative process. Such models are more likely to further our understanding about them and the biological drivers. With deeper understanding, novel architecture of higher complexity will be engineered. Furthermore, the learned statistical distribution is inevitably more meaningful and interpretable, facilitating applications in the healthcare domain and supporting the inference of insights. The aspects of privacy may become harder to prove in such cases.

\end{tcolorbox}
\hfill
\begin{tcolorbox}[tcbox width=auto, 
    arc=0mm, 
    colback=white, 
    coltext=cadmiumgreen, 
    boxrule=0pt, 
    colframe=white,
    box align=top]

\epigraph{Torture the data, and it will confess to anything.}{\textit{Ronald Coase}}

\end{tcolorbox}
\normalsize