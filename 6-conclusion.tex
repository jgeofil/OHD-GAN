\section{Conclusion}
\normalsize
\Gls{sd} has been a subject of interest for quite some time, with officials seeing enough value to launch longitudinal state-wide endeavours such as the Synthetic Data Project (SDP), funded by the United States Department of Education(USDOE) \cite{Bonnery2019-ug}. They dismiss a series of anonymization techniques, stating the burden on worker and financial resources, and the privacy guarantees that would not sufficient for governmental agencies. Issues that have only gained weight with the accumulation of big data, and the number of new sources growing consistently. The questions they hoped to answer at the start of the project in 2016 are still not fully answered (evaluation, scientific validity, legal implications). Their 2019 report on the experience is packed with interesting insights. Noting the distrust people tend to have of synthetic data, they were the ones who first proposed the idea conducting experiments on synthetic data, that could then be confirmed on real data by simply sending the analysis to the data holder (with the logistics described extensively and augmented by a flowchart).\par

The publication ends with a series case reports. The instances where the data could not satisfy requirements are analyzed with the aim of informing similar projects in the future. However the bulk of reports describe cases where was highly applicable. They concluded by predicting that the cost of generating \gls{sd} will diminish and that the methods to do so will improve. Their hopes for \gls{sd} include: easier access for researcher to the wealth of data, increased access providing downstream benefits at the state level, the these benefits encourage others to undertake similar projects that would increase generalizability of findings across states, and a preference for open data.

\renewcommand{\epigraphsize}{\footnotesize}
\setlength{\epigraphwidth}{12cm}
\epigraph{
    "[although some argue for] having secured data centers for administrative data utilization [...], our experience suggests that such centers may not solve the desire for fast turn-around research or broaden access to those with unique perspectives. Synthetic data represent a promising approach for increasing easy access to secure data while simultaneously protecting the confidentiality of individuals."}{\textit{Daniel Bonnéry, Yi Feng, Angela K. Henneberger, Tessa L. Johnson,\\ Mark Lachowicz, Bess A. Rose, Terry Shaw,\\ Laura M. Stapleton, Michael E. Woolley and Yating Zheng}}
    
The \gls{gan} was devised in 2014 in Montreal, Canada by \citeauthor{goodgan} at Université de Montreal. Two years before the start of the SDP, which must of been planned over a few years. It was too early for them to know about this obscure technique based on two neural networks competing against each other. Since then, \href{http://papers.nips.cc/paper/5423-generative-adversarial-nets}{Generative Adversarial Networks} \cite{goodgan} has inspired \textbf{23805} citations and algorithms capable of synthesizing data of impeccable similitude. We have surveyed a multitude of \gls{gan} algorithms built on the same basic idea of trial and error against an opponent that learns your faults. Despite the simple concept, we've seen that their range of application is wide, in general as well as in the healthcare domain. The variety of architectures and techniques we've seen reflect the heterogeneity of health data. Seemingly the difficulty of achieving stable learning with \glspl{gan} in general delayed their application to \gls{ohd}, while in medical imaging the development boomed much earlier sustained by the success of \gls{cnn} in other fields. Notably, the innovation in the field of \gls{cnn} has not slowed down after a few algorithms obtained excellent performance in image classification, but has deepened and branched out. Research concerning \gls{ohd} seems to be gaining momentum rapidly. We saw thoughtfully engineered algorithms designed for the characteristics of \gls{ohd}. Crucially however, pushing the research further will require a community effort to discover and defined metrics and standards upon which we can base objective assessment of models and \gls{sd}. The challenges posed by \gls{ohd} are nothing but encouragement for investigation and interpretation that can further our understanding of \glspl{gan}, machine learning and human health. Undoubtedly the innovations made for \gls{ohd} will find matches in other fields which may share the same data troubles.
\section{Acknowledgments}
This work was supported by KWF-PROTRAIT, NWO-TRAIN and a personal research grant from Stichting Hanarth Fonds. The authors would like to thank Sutit Nayak for her help with literature search.
\subsection{Author Contribution}
\noindent
Conceptualization, Methodology, literature search and review, Formal Analysis and Narrative, Writing – Original, second and third Draft, J.G.-F.\\ 
Expert Review, L.V., A.D.\\
Supervision, E.C\\
\section*{Conflict of interest}
The authors declare that there is no conflict of interest.
\pagebreak