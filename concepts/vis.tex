\begin{figure}
    \footnotesize
\noindent
\tcbsetforeverylayer{autoparskip}
\tcbset{enhanced, nobeforeafter, width=1\linewidth}
\begin{tcolorbox}[sidebyside,arc=0.5mm, 
    colback=MidnightBlue!10!white, 
    coltext=MidnightBlue!90!black,  
    colframe=MidnightBlue!90!black,
    colbacktitle=MidnightBlue!80,
    leftrule=0mm,
    rightrule=0mm, 
    toprule=0mm, 
    bottomrule=0mm,
    box align=top,
    title={\begin{panel}Representation and visualisation \label{pan:visualisation}\end{panel}}]

\citeauthor{Ledesma2016-hn} describe the problem of medical data representation and visualization thoughtfully, from information quality and usefulness, timescales and perception, to user satisfaction and aesthetics. The evaluation of their solution is extensive, detailed and rigorous, done according to the well known Nielsen's heuristics for Human-Computer Interaction \cite{nielsend}.  Interested readers can find the remainder here \href{https://www.nngroup.com/articles/ten-usability-heuristics/}{Ten Usability Heuristics.} While this may seem like a total digression towards graphic design, it is rather to illustrate the complexity of aspects to be considered before representing data in a evaluation task.
\tcblower

\textbf{Principle \#2: Match between system and the real world}:\\ \textit{The system should speak the users' language, with words, phrases and concepts familiar to the user, rather than system-oriented terms. Follow real-world conventions, making information appear in a natural and logical order. }
\\
---
\\
\textbf{Principle \#4 Consistency and standards}\\
\textit{Users should not have to wonder whether different words, situations, or actions mean the same thing. Follow platform conventions.
}

\end{tcolorbox}
\normalsize
\end{figure}